## Related Work

### Liquid Neural Networks
The concept of Liquid Neural Networks was first introduced by Hasani et al. \cite{hasani2021liquid}, 
who demonstrated that continuous-time neural networks based on ordinary differential equations (ODEs) 
could achieve superior performance with fewer parameters than traditional discrete-time networks. 
Their work established the theoretical foundation for adaptive computation graphs that dynamically 
adjust their temporal dynamics based on input complexity.

### Neural ODEs and Continuous Computation
Chen et al. \cite{chen2018neural} pioneered Neural Ordinary Differential Equations (NODEs), 
showing that residual networks could be interpreted as discrete approximations of continuous 
transformations. This work laid the groundwork for understanding neural networks as continuous 
dynamical systems, enabling memory-efficient backpropagation through time.

### Neuromorphic Audio Processing
Previous work in neuromorphic audio processing has focused primarily on spike-based neural 
networks for keyword spotting and voice activity detection. However, these approaches often 
require specialized hardware and have limited adaptability to varying signal conditions.

### Edge AI for Audio
The challenge of deploying neural networks on edge devices for audio processing has been 
extensively studied. Warden \cite{warden2018speech} established important benchmarks for 
keyword spotting on resource-constrained devices, while Sainath et al. \cite{sainath2015learning} 
demonstrated the effectiveness of CNN-LSTM architectures for raw audio processing.

### Our Contributions
Building upon this foundation, we introduce novel extensions to Liquid Neural Networks 
specifically optimized for ultra-low-power audio processing. Our key contributions include:
(1) Attention-driven adaptive timestep control, (2) Hierarchical liquid memory networks, 
(3) Quantum-classical hybrid dynamics, and (4) comprehensive power-accuracy optimization 
frameworks for edge deployment.